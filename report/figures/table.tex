\begin{table}[h]
\centering
\resizebox{\linewidth}{!}{%
\begin{tabular}{|l|l|p{5cm}|p{5cm}|}
\hline
\textbf{Method} & \textbf{Global Error} & \textbf{Pros} & \textbf{Cons} \\ \hline
Explicit Euler & $O(\Delta t)$ & Simple; low computational cost per step; easy to implement & Poor stability for larger $\Delta t$ in chaotic regimes; error accumulates linearly \\ \hline
RK4 & $O(\Delta t^4)$ & Higher accuracy and stability; better error control over fixed step sizes & More computationally expensive per step; increased complexity compared to Euler \\ \hline
Adaptive Methods (e.g., RK45) & Varies with adaptive step-size & Dynamically adjusts $\Delta t$ based on error; accurate and efficient in varying dynamics & More complex; computational overhead for step adjustment \\ \hline
\end{tabular}%
}
\caption{Comparison of Integration Methods}
\label{tab:integration_methods}
\end{table}


\begin{table}[h!]
    \centering
    \label{tab:diff_two_models_predator_prey}
    \begin{tabular}{|l|p{4.5cm}|p{4.5cm}|}
    \hline
    \textbf{Feature} & \textbf{Agent-Based Modeling (ABM)} & \textbf{Equation-Based Modeling (EBM)} \\
    \hline
    Modeling Units & Represents individual agents, such as each rabbit or fox, capturing their unique behaviors and interactions. & Models populations as continuous variables, focusing on aggregate properties rather than individual entities. \\
    \hline
    Stochasticity & Incorporates a high degree of randomness in agents' movements, births, deaths and so on. & Typically deterministic. \\
    \hline
    Spatiality & Explicitly models space, allowing agents to move and interact within an environment (in our case, a 2D environment). & Generally lacks spatial modeling. \\
    \hline
    Emergence & Enables emergent patterns from simple interaction rules among agents. & Patterns are outcomes of governing equations; less emphasis on emergence. \\
    \hline
    Complex Interactions & Models diverse and complex behaviors at the individual level. & Simplifies interactions by averaging behaviors across the population. \\
    \hline
    Realism & More biologically realistic, but computationally expensive. & Abstract, analytically tractable, and computationally less expensive, but may lack individual-level detail. \\
    \hline
    \end{tabular}
    \caption{Comparison of Agent-Based Modeling (ABM) and Equation-Based Modeling (EBM)}
\end{table}

\begin{table}[h!]
    \centering
    \begin{tabular}{lcccc}
    \hline
    \textbf{Variable} & \textbf{Mean} & \textbf{Std Dev} & \textbf{Min} & \textbf{Max} \\
    \hline
    Peak Rabbits     & 317.00 & 44.60  & 249   & 433  \\
    Min Rabbits      & 13.57  & 5.48   & 3     & 27   \\
    Final Rabbits    & 88.03  & 68.33  & 12    & 257  \\
    Peak Foxes       & 34.07  & 7.32   & 18    & 51   \\
    Min Foxes        & 4.33   & 2.29   & 0     & 10   \\
    Final Foxes      & 18.60  & 9.70   & 0     & 36   \\
    Peak Grass       & 5055.33 & 702.21 & 4307  & 7490 \\
    Min Grass        & 1122.37 & 141.13 & 775   & 1324 \\
    Final Grass      & 2646.97 & 887.27 & 1614  & 4444 \\
    \hline
    \end{tabular}
    \caption{
        Descriptive statistics of key ecological variables recorded across 100 independent simulation runs of the Agent-Based Model (ABM), with each run spanning up to 1000 iterations. The variables include the peak, minimum, and final values for the population of rabbits and foxes, as well as the amount of grass in the environment. These metrics provide insight into the dynamics of species interactions and resource availability, helping to evaluate ecosystem sustainability and identify scenarios that may lead to population collapse or resource depletion. See Figure~\ref{fig:boxplot_metrics_100_sim} for the distribution of these metrics.
    }
    \label{tab:simulation_stats_2.1}
\end{table}
